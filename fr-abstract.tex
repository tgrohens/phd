\selectlanguage{french}

\chapter{Résumé en français}

L'évolution des êtres vivants par sélection naturelle est souvent présentée comme un processus imprédictible, car elle trouve sa source dans les mutations aléatoires qui affectent le cœur du vivant, la molécule d'ADN dont la séquence est le support principal de l'information biologique.
Pourtant, s'il n'est pas possible de prévoir quelle mutation va survenir, et à quel endroit, de nombreuses expériences laissent penser que le chemin que suit l'évolution, c'est-à-dire les mutations qui sont observées, n'est pas totalement dû au hasard, et peut même être reproductible.
Si ce constat peut sembler évident à l'échelle de l'organisme, en pensant entre autres aux nombreuses plantes et animaux sélectionnés et domestiqués par l'espèce humaine depuis des dizaines de milliers d'années, ce n'est toutefois que depuis le XXe siècle que l'on est capable d'essayer d'en comprendre les soubassements à l'échelle moléculaire.
On observe alors que c'est parfois le même gène, voire le même nucléotide à l'intérieur d'un gène, qui est touché par des mutations lorsqu'on répète une expérience de sélection pour une caractéristique donnée ; l'évolution ne semble alors plus pouvoir suivre une multitude de chemins différents pour parvenir au même résultat phénotypique, mais contrainte de se tenir à un itinéraire bien défini.

Pendant ma thèse, je me suis consacré à l'étude d'un des mécanismes qui peuvent expliquer ce caractère répétable de l'évolution, à savoir l'épistasie, ou le rôle que joue le milieu génétique sur l'effet d'une mutation donnée.
En effet, il est possible qu'une mutation ait un effet favorable en présence d'une autre mutation, mais un effet défavorable en l'absence de celle-ci.
Ces relations épistatiques peuvent ainsi contraindre les options qui se présentent à l'évolution, en imposant qu'une mutation d'un gène donné survienne avant celle d'un second, afin que celle-ci soit favorable.
Le type de relations épistatiques le plus souvent étudié est celui des interactions entre mutations ponctuelles, c'est-à-dire entre mutations n'affectant qu'un faible nombre de nucléotides contigus à l'intérieur d'un même gène, car ce sont les mutations les plus faciles à détecter.
De nombreux autres types d'interactions épistatiques existent cependant. -> bancal non ?
La duplication d'un gène suivie d'une divergence et d'une spécialisation de ses deux copies dans deux fonctions différentes, processus jouant un rôle majeur dans l'innovation à l'échelle moléculaire, peut ainsi par exemple s'interpréter dans ce cadre.
Il y a alors épistasie entre un réarrangement chromosomique, la duplication d'un gène, et les mutations ponctuelles subséquentes ; en l'absence de cette duplication, les mutations qui rendent possible la spécialisation des copies du gènes seraient en effet délétères.

Le point de départ des travaux menés pendant ma thèse a donc été l'étude d'un cas particulier d'interactions épistatiques, celles engendrées par les mutations dans les gènes régulant la superhélicité de l'ADN bactérien.
La superhélicité de l'ADN, ou le niveau d'enroulement de l'ADN autour de lui-même, joue en effet un rôle important dans la régulation de l'activité des gènes bactériens, car le niveau de transcription des gènes dépend directement de la superhélicité au niveau de leur promoteur.
Comme le niveau de superhélicité est finement régulé par l'activité de plusieurs enzymes, appelées topoisomérases, une mutation dans un gène codant pour l'une de celles-ci peut engendrer un changement de l'activité transcriptionnelle à l'échelle du génome entier,
ouvrant alors la porte à l'émergence de nombreuses mutations compensatrices.
Le rôle évolutif des mutations de superhélicité, et leur caractère répétable, a en particulier été mis en exergue dans la \emph{Long Term Evolution Experiment} menée dans le laboratoire de Richard Lenski depuis 1988.

Afin de préciser quantitativement le rôle joué par ces mutations au cours de l'adaptation à un nouvel environnement, j'ai donc commencé par intégrer un modèle d'activité des gènes prenant en compte le niveau de superhélicité à l'échelle du chromosome dans un logiciel de simulation d'évolution existant au sein de mon équipe de thèse, le logiciel \emph{Aevol}.
Ce modèle, et les résultats obtenus à l'aide de celui-ci, sont présentés dans le chapitre~\ref{chap:aevol} de la thèse.
Les expériences menées dans ce cadre n'ont toutefois pas abouti à un résultat probant, la superhélicité convergeant très rapidement au cours de l'évolution à un niveau constant, alors que le reste du génome des individus continue d'évoluer.



\selectlanguage{english}