\selectlanguage{french}

\chapter{Résumé en français}

L'évolution des êtres vivants par sélection naturelle est souvent présentée comme un processus impossible à prédire, car elle trouve sa source dans les mutations aléatoires qui affectent le cœur du vivant: la molécule d'ADN, dont la séquence est le support principal de l'information biologique.
Pourtant, s'il n'est pas possible d'identifier avec certitude quelles mutations précises vont survenir en réponse à une pression de sélection donnée, de nombreuses expériences laissent penser que le chemin que suit l'évolution n'est pas entièrement dû au hasard et peut même être reproductible.
Ce constat peut sembler évident à l'échelle des organismes, en pensant entre autres aux nombreuses plantes et animaux sélectionnés et domestiqués par l'espèce humaine depuis des dizaines de milliers d'années, ou à l'apparition répétée de résistance aux traitements des infections bactériennes ou virales.
Ce n'est toutefois que depuis le XXe siècle que l'on est capable d'essayer de comprendre les soubassements de cette répétabilité à l'échelle moléculaire.
On observe alors que c'est parfois le même gène, voire le même nucléotide à l'intérieur d'un gène, qui est touché par des mutations lorsqu'on répète une expérience de sélection pour une caractéristique donnée.
Dans ce cas, l'évolution ne semble ainsi plus pouvoir suivre une multitude de chemins différents pour parvenir au même résultat visible, mais contrainte de se tenir à un itinéraire bien défini.

L'un des mécanismes qui peuvent expliquer ce caractère répétable de l'évolution est l'épistasie, ou le rôle que joue le milieu génétique sur l'effet d'une mutation donnée.
En effet, il est possible qu'une mutation ait un effet favorable en présence d'une autre mutation, mais un effet défavorable en l'absence de celle-ci.
Ces relations épistatiques peuvent ainsi contraindre les options qui se présentent à l'évolution, en imposant qu'une mutation d'un gène donné survienne avant celle d'un second gène, afin que celle-ci soit favorable.
Dans un contexte de compétition entre souches différentes au sein d'une même population (par exemple, de bactéries pathogènes), mieux comprendre ces relations épistatiques permettrait alors par exemple de prédire plus finement la fixation ou non de futures mutations, et par là la souche victorieuse, offrant la possibilité d'orienter plus finement un traitement.
Le type de relations épistatiques le plus souvent étudié est celui des interactions entre mutations ponctuelles, c'est-à-dire entre mutations n'affectant qu'un faible nombre de nucléotides contigus à l'intérieur d'un même gène, car ce sont les mutations les plus faciles à détecter.
Il existe toutefois de nombreux autres types d'interactions épistatiques, plus complexes et moins bien étudiés.
La duplication d'un gène suivie d'une divergence et d'une spécialisation de ses deux copies dans deux fonctions différentes, processus jouant un rôle majeur dans l'innovation à l'échelle moléculaire, peut ainsi par exemple s'interpréter dans ce cadre.
Il y a alors épistasie entre un réarrangement chromosomique, la duplication d'un gène et les mutations ponctuelles subséquentes ; en l'absence de cette duplication, les mutations qui rendent possible la spécialisation des copies du gènes seraient en effet délétères. \textbf{est-ce que l'exemple est vraiment nécessaire ?}

Le point de départ de ma thèse a donc été l'étude d'un cas particulier d'interactions épistatiques, celles engendrées par les mutations dans les gènes régulant la superhélicité de l'ADN.
La superhélicité de l'ADN, ou le niveau d'enroulement de l'ADN autour de lui-même, joue en effet un rôle important dans la régulation de l'activité des gènes chez les bactéries, car le niveau de transcription des gènes dépend directement de la superhélicité au niveau de leur promoteur.
Comme le niveau de superhélicité est finement régulé par l'activité de plusieurs enzymes, appelées topoisomérases, une mutation dans un gène codant pour l'une de celles-ci peut engendrer un changement de l'activité transcriptionnelle à l'échelle du génome entier,
ouvrant alors la porte à l'émergence de nombreuses mutations compensatrices.
Le rôle évolutif des mutations de superhélicité, et leur caractère répétable, a en particulier été mis en exergue dans la \emph{Long Term Evolution Experiment} menée dans le laboratoire de Richard Lenski, qui fait évoluer depuis 1988 12 souches d'\emph{Escherichia coli} dans un environnement de laboratoire.
En effet, 11 des 12 souches ont vu leur superhélicité augmenter très tôt dans le cours de l'expérience.

Pour aborder l'étude du rôle évolutif de ces mutations au cours de l'adaptation à un nouvel environnement, j'ai opté pour une approche de biologie évolutive des systèmes.
J'ai commencé par intégrer un modèle d'expression des gènes prenant en compte le niveau de superhélicité à l'échelle du chromosome dans un logiciel de simulation d'évolution existant au sein de mon équipe de thèse, le logiciel \emph{Aevol}.
Ce modèle, et les premiers résultats obtenus à l'aide de celui-ci, sont présentés dans le chapitre~\ref{chap:aevol} de la thèse.
Les expériences menées dans ce cadre n'ont toutefois pas abouti à un résultat probant, la superhélicité convergeant très rapidement au cours de l'évolution à un niveau constant, alors même que le reste du génome des individus continue d'évoluer.

Le rôle dans la régulation de l'expression des gènes que tient le niveau de superhélicité de l'ADN bactérien vient en réalité du caractère extrêmement dynamique de la superhélicité, qui est en grande partie lié à la transcription des gènes. \textbf{??}
Lorsqu'un gène est transcrit par une ARN polymérase, l'encombrant complexe enzymatique qui en résulte est incapable de tourner autour de l'ADN aussi vite que celui-ci ne tourne sur lui-même, ce qui génère une accumulation de superhélicité en aval du gène, et un déficit de superhélicité en amont de celui-ci.
La transcription d'un gène donné peut donc influer, par l'intermédiaire des changements de superhélicité dans son voisinage qu'elle engendre, sur la transcription des gènes à proximité de celui-ci, créant ainsi un réseau d'interactions et de couplages entre les niveaux d'expressions de gènes proches sur le génome.
Une modélisation aussi détaillée de la superhélicité se révélant délicate à implémenter dans le modèle \emph{Aevol}, j'ai opté pour l'étude, dans un premier temps, de son rôle évolutif dans un modèle intégrant un génome simplifié, mais décrivant plus fidèlement la superhélicité.
J'ai implémenté ce modèle, appelé \emph{EvoTSC}, à partir de zéro en Python, et il est disponible à l'adresse \url{https://gitlab.inria.fr/tgrohens/evotsc}.

À l'aide d'\emph{EvoTSC}, j'ai dans un premier temps montré que, dans un modèle où le seul mécanisme de régulation de l'activité des gènes est le couplage médié par la superhélicité entre les niveaux de transcription de gènes proches, et où les seules mutations possibles sont les inversions chromosomiques (qui réorganisent les positions relatives des gènes), il est possible d'obtenir par sélection naturelle des individus dont les gènes suivent avec précision des cibles dépendant de l'environnement.
En particulier, il est possible d'obtenir des gènes activés par la relaxation globale de l'ADN, alors que leurs promoteurs sont inhibés par une relaxation locale.
Ces premiers résultats sont présentés dans le chapitre~\ref{chap:alife}, et démontrent que la superhélicité peut bien jouer un rôle majeur dans la régulation de l'activité des gènes bactériens, en tant que support d'un réseau de régulation génétique.
Ils ont été publiés, d'abord sous forme d'article dans la conférence \emph{ALIFE 2021}~\citep{grohens2021}, puis dans une version étendue dans le journal associé, \emph{Artificial Life}~\citep{grohens2022a}.

Dans un second temps, j'ai ensuite cherché à caractériser plus en détail l'impact évolutif de la superhélicité dans la structure des génomes bactériens.
Toujours en utilisant le modèle \emph{EvoTSC}, j'ai montré qu'au niveau le plus local, des paires convergentes ou divergentes de gènes voisins se forment, conformément aux prédictions théoriques du couplage entre superhélicité et transcription.
J'ai montré que cette organisation à l'échelle locale du génome n'était toutefois pas entièrement suffisante pour expliquer les niveaux d'expression des gènes observés dans le génome complet, mais que des sous-réseaux impliquant jusqu'à plusieurs dizaines de gènes peuvent au contraire être nécessaires.
Enfin, en utilisant une approche par KO de gène, j'ai montré que dans les génomes des individus évolués, c'est sous la forme d'un réseau unique, s'étendant à l'échelle du génome entier, que s'organise la régulation de l'expression des gènes dans le modèle \emph{EvoTSC}.
Ce second ensemble de résultats est présenté dans le chapitre~\ref{chap:ploscb}, et a été mis en forme dans une prépublication~\citep{grohens2022b}, prochainement soumise à relecture par les pairs.

Dans le chapitre~\ref{chap:params}, je présente ensuite un ensemble d'expériences complémentaires qui montrent la robustesse des résultats du modèle \emph{EvoTSC} présentés jusqu'ici, lorsque l'on fait varier les principaux paramètres du modèle.
J'ai finalement incorporé dans \emph{EvoTSC} un modèle d'évolution du niveau de superhélicité globale, afin de pouvoir caractériser comme dans les expériences menées avec \emph{Aevol} les possibles relations épistatiques entre mutations de superhélicité et réarrangement chromosomiques.
Ces résultats sont présentés dans le chapitre~\ref{chap:epistasis}.

L'annexe~\ref{chap:software} présente les contributions logicielles que j'ai réalisées tout au long de ma thèse.
J'ai d'abord participé au développement d'\emph{Aevol} et d'outils associés pour gérer des simulations, puis développé le modèle \emph{EvoTSC} ainsi qu'un ensemble d'outils pour visualiser et analyser les données en résultant.

Pour finir, le cours de ma thèse ayant été perturbé par l'irruption de la pandémie de Covid-19 en France au printemps 2020, je me suis porté volontaire pour collaborer, au sein d'une équipe de chercheur·es et d'ingénieur·es Inria, avec l'Assistance Publique-Hôpitaux de Paris (AP-HP).
Nous avons ensemble construit un modèle de l'épidémie dans l'agglomération parisienne, qui visait à aider les équipes de l'AP-HP à suivre en temps réel et essayer de prédire l'évolution de l'épidémie à l'aide de données de régulation médicale.
Ces travaux ont par la suite mené à une publication~\citep{gaubert2020}, présentée dans l'annexe~\ref{chap:covid}.

\selectlanguage{english}
