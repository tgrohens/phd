\selectlanguage{french}

\chapter{Résumé en français}

L'évolution des êtres vivants par sélection naturelle est souvent présentée comme un processus imprédictible, car elle trouve sa source dans les mutations aléatoires qui affectent le cœur du vivant, la molécule d'ADN dont la séquence est le support principal de l'information biologique.
Pourtant, s'il n'est pas possible de prévoir quelle mutation va survenir, et à quel endroit, de nombreuses expériences laissent penser que le chemin que suit l'évolution, c'est-à-dire les mutations qui sont observées, n'est pas totalement dû au hasard, et peut même être reproductible.
Si ce constat peut sembler évident à l'échelle de l'organisme, en pensant entre autres aux nombreuses plantes et animaux sélectionnés et domestiqués par l'espèce humaine depuis des dizaines de milliers d'années, ce n'est toutefois que depuis le XXe siècle que l'on est capable d'essayer d'en comprendre les soubassements à l'échelle moléculaire.
On observe alors que c'est parfois le même gène, voire le même nucléotide à l'intérieur d'un gène, qui est touché par des mutations lorsqu'on répète une expérience de sélection pour une caractéristique donnée ; l'évolution ne semble alors plus pouvoir suivre une multitude de chemins différents pour parvenir au même résultat phénotypique, mais contrainte de se tenir à un itinéraire bien défini.

Pendant ma thèse, je me suis consacré à l'étude d'un des mécanismes qui peuvent expliquer ce caractère répétable de l'évolution, à savoir l'épistasie, ou le rôle que joue le milieu génétique sur l'effet d'une mutation donnée.
En effet, il est possible qu'une mutation ait un effet favorable en présence d'une autre mutation, mais un effet défavorable en l'absence de celle-ci.
Ces relations épistatiques peuvent ainsi contraindre les options qui se présentent à l'évolution, en imposant qu'une mutation d'un gène donné survienne avant celle d'un second gène, afin que celle-ci soit favorable.
Le type de relations épistatiques le plus souvent étudié est celui des interactions entre mutations ponctuelles, c'est-à-dire entre mutations n'affectant qu'un faible nombre de nucléotides contigus à l'intérieur d'un même gène, car ce sont les mutations les plus faciles à détecter.
De nombreux autres types d'interactions épistatiques existent cependant. \textbf{-> bancal non ?}
La duplication d'un gène suivie d'une divergence et d'une spécialisation de ses deux copies dans deux fonctions différentes, processus jouant un rôle majeur dans l'innovation à l'échelle moléculaire, peut ainsi par exemple s'interpréter dans ce cadre.
Il y a alors épistasie entre un réarrangement chromosomique, la duplication d'un gène, et les mutations ponctuelles subséquentes ; en l'absence de cette duplication, les mutations qui rendent possible la spécialisation des copies du gènes seraient en effet délétères.

\paragraph{}
Le point de départ des travaux menés pendant ma thèse a donc été l'étude d'un cas particulier d'interactions épistatiques, celles engendrées par les mutations dans les gènes régulant la superhélicité de l'ADN bactérien.
La superhélicité de l'ADN, ou le niveau d'enroulement de l'ADN autour de lui-même, joue en effet un rôle important dans la régulation de l'activité des gènes chez les bactéries, car le niveau de transcription des gènes dépend directement de la superhélicité au niveau de leur promoteur.
Comme le niveau de superhélicité est finement régulé par l'activité de plusieurs enzymes, appelées topoisomérases, une mutation dans un gène codant pour l'une de celles-ci peut engendrer un changement de l'activité transcriptionnelle à l'échelle du génome entier,
ouvrant alors la porte à l'émergence de nombreuses mutations compensatrices.
Le rôle évolutif des mutations de superhélicité, et leur caractère répétable, a en particulier été mis en exergue dans la \emph{Long Term Evolution Experiment} menée dans le laboratoire de Richard Lenski depuis 1988.

Afin de préciser quantitativement le rôle joué par ces mutations au cours de l'adaptation à un nouvel environnement, j'ai donc commencé par intégrer un modèle d'expression des gènes prenant en compte le niveau de superhélicité à l'échelle du chromosome dans un logiciel de simulation d'évolution existant au sein de mon équipe de thèse, le logiciel \emph{Aevol}.
Ce modèle, et les premiers résultats obtenus à l'aide de celui-ci, sont présentés dans le chapitre~\ref{chap:aevol} de la thèse.
Les expériences menées dans ce cadre n'ont toutefois pas abouti à un résultat probant, la superhélicité convergeant très rapidement au cours de l'évolution à un niveau constant, alors même que le reste du génome des individus continue d'évoluer.

Le rôle dans la régulation de l'expression des gènes que tient le niveau de superhélicité de l'ADN bactérien vient en réalité de son caractère extrêmement dynamique, celui-ci variant non seulement le long du génome mais aussi pendant le cycle de vie de la bactérie, alternant entre phases de croissance exponentielle et phases stationnaire.
Une modélisation aussi détaillée de la superhélicité se révélant délicate à implémenter dans le modèle \emph{Aevol}, j'ai donc opté pour l'étude, dans un premier temps, de son rôle évolutif dans un modèle simplifié en termes de génome mais plus précis pour ce qui est de la superhélicité, que j'ai implémenté à partir de zéro en Python: le modèle \emph{EvoTSC}, disponible à l'adresse \url{https://gitlab.inria.fr/tgrohens/evotsc}.
Si la superhélicité est une caractéristique aussi variable du chromosome bactérien, c'est en effet en partie car elle est sensible à la transcription des gènes situés sur celui-ci.
Lorsqu'un gène est transcrit par une ARN polymérase, cet encombrant complexe enzymatique est incapable de tourner autour de l'ADN à la même vitesse que celui-ci tourne sur lui-même, ce qui génère une accumulation d'enroulements, et donc de superhélicité, en aval du gène, et un déficit de superhélicité en amont de celui-ci.
Dans les génomes bactériens, les variations de superhélicité se propagent le long du génome quasi-instantanément par rapport à la vitesse de transcription des ARN polymérases, et à une distance qui est comparable à la distance qui sépare un gène de ses voisins.
Ainsi, la transcription d'un gène donné peut influer, par l'intermédiaire des changements de superhélicité dans son voisinage qu'elle engendre, sur la transcription des gènes à proximité de celui-ci, créant donc un possible réseau d'interactions et de couplages entre les niveaux d'expressions de gènes proches.

À l'aide du modèle \emph{EvoTSC}, j'ai dans un premier temps montré que, dans un modèle où le seul mécanisme de régulation de l'activité des gènes est le couplage, médié par la superhélicité, entre les niveaux de transcription (et donc d'expression) de gènes proches sur le génome, et où les seules mutations possibles sont les inversions chromosomiques, qui peuvent réorganiser les positions relatives des gènes, il est possible d'obtenir par sélection naturelle des individus dont les gènes suivent avec précision des cibles qui dépendent de l'environnement.
Ces premiers résultats sont présentés dans le chapitre~\ref{chap:alife}, et démontrent que la superhélicité peut bien jouer un rôle majeur dans la régulation de l'activité des gènes bactériens, en tant que support d'un réseau de régulation génétique.
Ils ont été publiés, d'abord sous forme d'article dans la conférence \emph{ALIFE 2021}~\citep{grohens2021}, puis dans une version étendue dans le journal associé, \emph{Artificial Life}~\citep{grohens2021}. (\textbf{corriger la citation})

Dans un second temps, j'ai ensuite cherché à caractériser plus en détail l'impact évolutif de la superhélicité dans la structure des génomes bactériens, en m'intéressant à la disposition des gènes sur le génome résultant des nombreux réarrangements qui forgent, au cours de l'évolution, le réseau de régulation précédemment évoqué.
Toujours en utilisant le modèle \emph{EvoTSC}, j'ai montré qu'au niveau le plus local, des paires de gènes voisins se forment, qui se positionnent conformément aux prédictions théoriques du couplage entre superhélicité et transcription afin de présenter plusieurs types d'interaction: soit en adoptant des orientations divergentes, afin de former une boucle de rétroaction positive entre les niveaux d'expression de gènes devant être actifs quel que soit l'environnement, soit au contraire en adoptant des orientations convergentes, formant une boucle de rétroaction négative qui résulte en un système bistable, pour des gènes devant être actifs dans des environnements distincts.
J'ai montré que cette organisation à l'échelle locale du génome n'était toutefois pas entièrement suffisante pour expliquer les niveaux d'expression des gènes observés dans le génome entier, mais que des sous-réseaux impliquant jusqu'à plusieurs dizaines de gènes pouvaient être nécessaires pour ce faire.
Enfin, en utilisant une approche par KO de gène, j'ai montré que dans les génomes des individus évolués, c'est sous la forme d'un réseau unique, s'étendant à l'échelle du génome entier, que s'organise la régulation de l'expression des gènes dans le modèle \emph{EvoTSC}.
Ce second ensemble de résultats est présenté dans le chapitre~\ref{chap:ploscb}, et a été rédigé dans une prépublication~\citep{grohens2021} \textbf{changer la citation}.

Enfin, le cours de ma thèse ayant été perturbé par l'irruption de la pandémie de Covid-19 en France au printemps 2020, j'ai été amené (sur la base du volontariat) à collaborer, au sein d'une équipe de chercheur·es et d'ingénieur·es Inria, avec l'Assistance Publique-Hôpitaux de Paris (AP-HP), afin d'aider leurs équipes à suivre en temps réel, et essayer de prédire, l'évolution de l'épidémie au sein de l'agglomération parisienne, à l'aide de données de régulation médicale.
Ces travaux ont par la suite mené à une publication~\citep{gaubert2020}, présentée dans l'annexe~\ref{chap:covid}.

\selectlanguage{english}
