\selectlanguage{french}

\chapter{Remerciements}

Le chemin qui mène à la thèse est long et sinueux, et il n'est pas de ceux que l'on parcourt seul.
J'aimerais donc remercier ici toutes celles et ceux qui m'ont accompagné au fil de celui-ci.

Tout d'abord, mon directeur de thèse, Guillaume Beslon.
J'ai, dès mon stage de licence puis tout au long de ma thèse, pris un grand plaisir à travailler sous sa bienveillante et motivante supervision.
Plus d'une question mal formulée, ou d'une intuition nébuleuse, se sont éclaircies grâce à lui pendant nos nombreuses discussions, et je le remercie de m'avoir ainsi aidé à naviguer au mieux les eaux hasardeuses de la recherche.
Je voudrais aussi le remercier pour sa disponibilité, sur laquelle j'ai toujours pu compter quand j'en ai eu le besoin.

Je remercie également Ivan Junier et Céline Scornavacca, qui m'ont fait l'honneur de rapporter ma thèse, et Guillaume Achaz, Nelle Varoquaux, Sam Meyer, et William Nasser d'avoir participé à mon jury ; je garderai longtemps le souvenir de ma soutenance.

Le doctorat est l'aboutissement d'un parcours qui ne saurait se résumer à ses dernières années, et je souhaiterais également remercier l'ensemble des professeur·es et enseignant·es qui m'ont transmis leurs savoirs au fil des ans et des disciplines.
J'ai une pensée particulière pour mon professeur de SVT de première, M. Piolet, qui m'a fait découvrir que la biologie pouvait se voir comme une énigme que l'on prend plaisir à résoudre.

Plus largement, je tiens à remercier tous les membres de l'équipe Beagle -- Guillaume, Hugues, Carole, Jonathan, Éric, Christophe, Anton, Leo, David -- pour le cadre de travail exceptionnel que vous avez construit, et qui m'a permis de m'épanouir pleinement ces trois dernières années.
Je souhaite aussi particulièrement remercier le personnel administratif et de soutien à la recherche, en premier lieu Laëtitia et Salwa, qui nous permettent de travailler dans les meilleures conditions possibles.
J'ai enfin une pensée pour la \emph{Doua en Lutte}, collectif avec lequel nous avons milité sur le campus contre la réforme des retraites (version 2019) et autres \emph{LPR}, et tout·es ses membres : j'espère que ce que nous avons fait ensemble n'aura pas été en vain.

Un immense merci à tout·es les doctorant·es, de cette équipe ou d'une autre, et aux autres collègues que j'ai eu le plaisir de côtoyer : Victor et sa maîtrise de l'art du café, Audrey, Vincent, Laurent, Marco et Julie avec qui j'ai partagé bien plus qu'un simple bureau, Paul, Lisa B. et Lisa C., Nathan, Ju', Hana, Charlotte, Arsène, Arnaud, Adrien, Aurélien, Mirabelle, Nicolas, et j'en oublie sans doute.
Que ce soit à la salle d'escalade, autour d'une bière à l'inénarrable \emph{Jeudi des Génies} de Leo, au cinéma ou ailleurs, ce sont tous ces moments en votre compagnie qui m'ont rendu ces années lyonnaises si agréables.

Je voudrais ensuite exprimer ma profonde affection et ma gratitude envers tout·es mes ami·es : Ulysse, pour son soutien indéfectible et son amour de Lyon, Matthieu et Philémon, avec qui j'ai le bonheur de cheminer depuis si longtemps, Jean et Solène pour ces studieuses journées en Salle 5, Auguste, Marie, Arthur, Clara, Ramzy, Camille, Gaspar et Achille, très chèr·es camarades de Courô, Sara, Amalia, Pierre, Mathilde, Pierre-Emmanuel, Sébastien, Martin, les deux Alex et la chouette communauté du Discord \emph{PhD Students} devenu \emph{IRL Lyon Ultimate + Shiny Edition}, et tout·es celleux que j'oublie ici.
Que ce soit en bibli ou au \emph{Melo}, à la montagne ou au bord de l'eau, à pied ou à vélo, le temps que je passe avec vous m'est infiniment précieux.
J'aimerais remercier encore une fois Matthieu et Marie, pour avoir eu la gentillesse de relire ce manuscrit avec un œil alerte et une alacrité sans égale.

Je remercie profondément Marie et sa famille de m'avoir accueilli dans leur foyer pendant cette si particulière période du premier confinement, Solène pour nous avoir offert une bienvenue fugue aux couleurs du granit rose et au parfum de sarrasin, et ma mère chez qui j'ai aussi pu me reposer avant de repartir à l'aventure à Lyon.

Merci Vinciane d'ensoleiller mes journées par ton enthousiasme et ta bonne humeur si communicatives, de m'avoir fait découvrir Copenhague et d'avoir exploré avec moi les docks de Strasbourg comme de Hambourg, et de partager avec moi la quête du meilleur restaurant de Lyon et tant d'autres choses encore.
Ta présence à mes côtés m'est chère.

J'aimerais enfin remercier mes parents, mon frère et ma sœur, qui me soutiennent depuis toujours et m'ont permis d'être qui je suis.

\paragraph{}À vous tout·es, infiniment merci.

\selectlanguage{english}
