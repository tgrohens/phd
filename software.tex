\chapter{Software Contributions}
\label{chap:soft}

\section{\emph{Aevol}}

For the first year of my Ph.D., I mainly used the \emph{Aevol} software platform to run experiments, which is available here: \url{https://gitlab.inria.fr/aevol/aevol}.
\emph{Aevol} has been in development in the Inria BEAGLE team for over 17 years, and contains over 90,000 lines of C++ code.
While I was using \emph{Aevol}, I took an active part in the maintenance of its complex code base, and particularly focused on fixing memory leaks, and on improving the overall quality of the code.
The associated commits can be found here: \url{https://gitlab.inria.fr/aevol/aevol/-/commits/main?author=Théotime%20Grohens}.

\subsection{DNA Supercoiling in \emph{Aevol}}

As presented in Section~\ref{sec:aevol:aevol_sc}, I developed a version of \emph{Aevol} that incorporates the effect of supercoiling on gene transcription into the model.
This version can be found here: \url{https://gitlab.inria.fr/tgrohens/aevol/-/tree/rebased-supercoiling}.
Developing this extension of the model required extensive changes in the evaluation of individuals, in order to take supercoiling into account; in the code handling the mutations, as I added a new kind of mutations to the model; and in the post-treatments, the code that analyses experimental data after the main simulation is complete.

\subsection{Tooling for \emph{Aevol}}

While using \emph{Aevol}, I developed a set of Python tools, \emph{aevol-utilities} (available here: \url{https://gitlab.inria.fr/tgrohens/aevol-utilities}), in order to make it easier to carry out full-fledged experiments.
The default \emph{Aevol} executables indeed do not provide the most accessible interface to an untrained user, and do not reflect the way the platform is used to run experiments at present.
The typical use case of \emph{Aevol} consists in the following steps, for each replicate of a given experiment:
\begin{enumerate}
  \item Create an initial random individual and initial population using this individual, with a different seed.
  \item Run the proper evolution experiment for a given number of generations.
  \item Pick an individual in the final population, and trace its lineage since the original population.
  \item Compute a series of statistics over this lineage, by recovering every individual in the lineage.
  \item Analyse and plot the resulting data.
\end{enumerate}

The \emph{aevol-utilities} package provide a Jupyter notebook which automatizes the first four steps of this pipeline, and that relies on a Python backend.
The tool takes care of creating properly initialized repositories in order to minimize the risk of data loss.
As \emph{Aevol} is computationally heavy, the tool also allows the user to choose the level of parallelism to use, and regulates the number of concurrently running replicates accordingly.
Finally, the package also contains Jupyter notebooks that give examples of how to analyse \emph{Aevol} data, that are easily approachable and modifiable.

\subsection{Web Version of \emph{Aevol}}


\section{\emph{EvoTSC}}
