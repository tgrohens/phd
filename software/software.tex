\chapter{Software Contributions}
\label{chap:software}

\section{\emph{Aevol}}

For the first year of my Ph.D., I mainly used the \emph{Aevol} \emph{in silico} experimental evolution platform, which is available here: \url{https://gitlab.inria.fr/aevol/aevol}, to run experiments.
\emph{Aevol} has been in development in the Inria BEAGLE team for over 17 years, and contains over 90,000 lines of C++ code.
It has been used to run experiments that resulted in numerous publications, among which~\cite{knibbe2005}, \cite{batut2013}, or \cite{rutten2019}.
While I was using \emph{Aevol}, I took an active part in the maintenance of its complex code base, and particularly focused on fixing memory leaks, and on improving the overall quality of the code.
The associated commits can be found here: \url{https://gitlab.inria.fr/aevol/aevol/-/commits/main?author=Théotime%20Grohens}.

\subsection{DNA Supercoiling in \emph{Aevol}}

As presented in Section~\ref{sec:aevol:aevol_sc}, I developed during my Ph.D. a version of \emph{Aevol} that incorporates the effect of supercoiling on gene transcription.
This version can be found here: \url{https://gitlab.inria.fr/tgrohens/aevol/-/tree/rebased-supercoiling}.
Developing this extension of the model required extensive changes in the evaluation of individuals, in order to take supercoiling into account; in the code handling the mutations, as I added a new kind of mutations to the model; and in the post-treatments, the code that analyses experimental data after the main simulation is complete.

\subsection{Tooling for \emph{Aevol}}

While using \emph{Aevol}, I developed a set of Python tools, \emph{aevol-utilities} (available here: \url{https://gitlab.inria.fr/tgrohens/aevol-utilities}), in order to make it easier to carry out full-fledged experiments.
The default \emph{Aevol} executables indeed do not provide the most accessible interface to an untrained user, and do not reflect the way the platform is used to run experiments at present.
The typical use case of \emph{Aevol} consists in the following steps, for each replicate of a given experiment:
\begin{enumerate}
  \item Create an initial random individual and initial population using this individual, with a different seed.
  \item Run the proper evolution experiment for a given number of generations.
  \item Pick an individual in the final population, and trace its lineage since the original population.
  \item Compute a series of statistics over this lineage, by recovering every individual in the lineage.
  \item Analyse and plot the resulting data.
\end{enumerate}

The \emph{aevol-utilities} repository comprises a Python library (named \texttt{aevol.py}), and a series of Jupyter notebooks that are built upon this library.
The \texttt{Runner aevol} notebook automates the first four steps of the pipeline above.
It takes care of creating properly initialized repositories in order to minimize the risk of data loss, of starting concurrents \emph{Aevol} jobs with a parametrizable level of parallelism (cores per job and number of parallel jobs), can handle restarting interrupted simulations, and can run \emph{Aevol} post-treatments once the main simulation is finished.
The package also contains Jupyter notebooks that give easily approachable and modifiable examples of how to analyse \emph{Aevol} data.
Finally, the \emph{aevol-utilities} tools are in use by other members of the team, and could be part of a larger move towards a new high-level Python interface for \emph{Aevol}.

%\subsection{Web Version of \emph{Aevol}}
% Ça a l'air de difficilement marcher...


\section{\emph{EvoTSC}}

In the second part of my Ph.D., I developed \emph{EvoTSC}, a Python simulation that implements an individual-level model of the transcription-supercoiling coupling and a population-level \emph{in silico} artificial evolution platform, which is accessible here: \url{https://gitlab.inria.fr/tgrohens/evotsc}.
Chapter~\ref{chap:alife} presents a first version of the transcription-supercoiling coupling model and the results obtained using a corresponding version of \emph{EvoTSC}, which were published in~\cite{grohens2021} and in~\cite{grohens2022a}.
The corresponding versions of the code, and accompanying data analysis notebooks, can respectively be found in the \href{https://gitlab.inria.fr/tgrohens/evotsc/-/tree/alife-model}{\texttt{alife-model}} and \href{https://gitlab.inria.fr/tgrohens/evotsc/-/tree/alife-journal}{\texttt{alife-journal}} branches of the git repository.

The current version of \emph{EvoTSC} implements the more detailed genome model presented in Section~\ref{sec:ploscb:model}, and is the one that is used in the results presented in Chapters~\ref{chap:ploscb} \textbf{ajouter les autres chapitres}.

\subsection{Technical Description}

\emph{EvoTSC} is written in Python, and measures around 1,800 lines of code.
The code runs an evaluation-selection-mutation evolutionary loop on a populations of individuals consisting in a circular string-of-pearls genome.
The computationally heavy parts are accelerated using \texttt{numba}~\citep{lam2015}.

%\lstinputlisting[language=Python, caption=Code of the \texttt{numba}-accelerated \texttt{compute\_inter\_matrix} method.]{software/evotsc_core.py}

\subsection{Tooling}

In order to analyze data obtained with the core \emph{EvoTSC} simulation more easily, I developed an extensive set of visualization tools, which are used in the figures presented throughout the manuscript.
In particular, I developed an extremely customizable view of the genomes, which can accommodate plotting the level of local supercoiling inside the genome (as in Figure \textbf{mettre une figure}), plotting subnetworks (\textbf{same}), or gene knockouts (\textbf{same}).


%\lstinputlisting[language=Python, caption=dkozkdz]{software/evotsc_core.py}
