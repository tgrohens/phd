\chapter{Software Contributions}
\label{chap:software}

\section{\emph{Aevol}}

For the first year of my Ph.D., I mainly used the \emph{Aevol} \emph{in silico} experimental evolution platform, which is available here: \url{https://gitlab.inria.fr/aevol/aevol}, to run experiments.
\emph{Aevol} has been in development in the Inria BEAGLE team for over 17 years, and contains over 90,000 lines of C++ code.
It has been used to run experiments that resulted in numerous publications, among which~\cite{knibbe2005}, \cite{batut2013}, or \cite{rutten2019}.
While I was using \emph{Aevol}, I took an active part in the maintenance of its complex code base, and particularly focused on fixing memory leaks, and on improving the overall quality of the code.
The associated commits can be found here: \url{https://gitlab.inria.fr/aevol/aevol/-/commits/main?author=Théotime%20Grohens}.

\subsection{DNA Supercoiling in \emph{Aevol}}

As presented in Section~\ref{sec:aevol:aevol_sc}, I developed during my Ph.D. a version of \emph{Aevol} that incorporates the effect of supercoiling on gene transcription.
This version can be found here: \url{https://gitlab.inria.fr/tgrohens/aevol/-/tree/rebased-supercoiling}.
Developing this extension of the model required extensive changes in the evaluation of individuals, in order to take supercoiling into account; in the code handling the mutations, as I added a new kind of mutations to the model; and in the post-treatments, the code that analyses experimental data after the main simulation is complete.

\subsection{Tooling for \emph{Aevol}}

While using \emph{Aevol}, I developed a set of Python tools, \emph{aevol-utilities} (available here: \url{https://gitlab.inria.fr/tgrohens/aevol-utilities}), in order to make it easier to carry out full-fledged experiments.
The default \emph{Aevol} executables indeed do not provide the most accessible interface to an untrained user, and do not reflect the way the platform is used to run experiments at present.
The typical use case of \emph{Aevol} consists in the following steps, for each replicate of a given experiment:
\begin{enumerate}
  \item Create an initial random individual and initial population using this individual, with a different seed.
  \item Run the proper evolution experiment for a given number of generations.
  \item Pick an individual in the final population, and trace its lineage since the original population.
  \item Compute a series of statistics over this lineage, by recovering every individual in the lineage.
  \item Analyse and plot the resulting data.
\end{enumerate}

The \emph{aevol-utilities} repository comprises a Python library (named \texttt{aevol.py}), and a series of Jupyter notebooks that are built upon this library.
The \texttt{Runner aevol} notebook automates the first four steps of the pipeline above.
It takes care of creating properly initialized repositories in order to minimize the risk of data loss, of starting concurrents \emph{Aevol} jobs with a parametrizable level of parallelism (cores per job and number of parallel jobs), can handle restarting interrupted simulations, and can run \emph{Aevol} post-treatments once the main simulation is finished.
The package also contains Jupyter notebooks that give easily approachable and modifiable examples of how to analyse \emph{Aevol} data.
Finally, the \emph{aevol-utilities} tools are in use by other members of the team, and could be part of a larger move towards a new high-level Python interface for \emph{Aevol}.

%\subsection{Web Version of \emph{Aevol}}
% Ça a l'air de difficilement marcher...


\section{\emph{EvoTSC}}

In the second part of my Ph.D., I developed \emph{EvoTSC}, a Python simulation that implements an individual-level model of the transcription-supercoiling coupling and a population-level \emph{in silico} artificial evolution platform.
The software is accessible here: \url{https://gitlab.inria.fr/tgrohens/evotsc}.
\emph{EvoTSC} has been used in several publications.
Chapter~\ref{chap:alife}, based on~\cite{grohens2021} and on~\cite{grohens2022a}, presents a first version of the transcription-supercoiling coupling model and the results obtained using a corresponding version of \emph{EvoTSC}.
Chapter~\ref{chap:ploscb} presents a more realistic version of the individual-level model and the associated results, which have been published as a preprint~\citep{grohens2022b} and will be submitted to peer review.
The corresponding versions of the code, and accompanying data analysis notebooks, can be found in the \href{https://gitlab.inria.fr/tgrohens/evotsc/-/tree/alife-model}{\texttt{alife-model}} \citep{grohens2021} and \href{https://gitlab.inria.fr/tgrohens/evotsc/-/tree/alife-journal}{\texttt{alife-journal}} \citep{grohens2022a} branches of the git repository.

The current version of \emph{EvoTSC} implements the more detailed genome model presented in Section~\ref{sec:ploscb:model}, and is the one that is used in the results presented in Chapters~\ref{chap:ploscb} and~\ref{chap:param}.

\subsection{Technical Description}

\emph{EvoTSC} is written in Python, and measures around 1,800 lines of Python code.
It comes as an installable \texttt{pip} package, and is licensed under a 3-clause BSD license.
The \texttt{evotsc.py} file contains the main classes (\texttt{Gene}, \texttt{Individual} and \texttt{Population}) that are used in the simulations.
The \texttt{core.py} files contains the computationally heavy parts of the code, which are accelerated using \texttt{numba}~\citep{lam2015}.
The \texttt{run.py} contains the code responsible for initializing a simulation, and data input and output.
The \texttt{lib.py} contains miscellaneous functions used throughout the notebooks.
Finally, the \texttt{plot.py} file contains the code responsible for plotting, using the \texttt{matplotlib} library~\citep{hunter2007}.
This file contains in particular the \texttt{plot\_genome\_and\_tsc} function used to create all the circular genome plots in the manuscript.

\subsection{Tooling}

In order to analyze the results of  \emph{EvoTSC} simulations more easily, I developed an extensive set of notebooks for data analysis and visualisation.
The notebooks are available in a separate repository: \url{https://gitlab.inria.fr/tgrohens/evotsc-notebooks}, and are also licensed under the BSD 3-clause.
In particular, these notebooks contain the code responsible for the computation of gene expression as a function of supercoiling (such as in Figure~\ref{fig:ploscb:activity_by_sigma}), for the generation of contiguous gene subnetworks (Figure~\ref{fig:ploscb:min_subnetwork}), and for the gene knock-out analysis (Figure~\ref{fig:ploscb:ko_genomes},~\ref{fig:ploscb:ko_graph} and~\ref{fig:ploscb:ko_graph}).

\subsection{Use}

I have up until now been the only user of \emph{EvoTSC}.
However, the code has been written to be readily reusable and extendable.
It is in particular documented with a \texttt{README.md} file, and extensive comments.


%\lstinputlisting[language=Python, caption=dkozkdz]{software/evotsc_core.py}
